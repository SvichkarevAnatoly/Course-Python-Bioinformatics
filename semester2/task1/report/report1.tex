\documentclass{article} % Класс печатного документа

\usepackage{hyperref} % Для вставки гиперссылок
\usepackage{listings} % Для вставки кусков кода
\usepackage{graphicx} % Вставка изображений
\usepackage[utf8]{inputenc} % Кодировка исходного текста - utf8
\usepackage[english,russian]{babel} % Поддержка языка - русского с английским
\usepackage{indentfirst} % Отступ в первом абзаце

\title{Отчёт 8\protect\\Интерфейс с <<R>> для решения статистических задач} % Заголовок документа
\author{Свичкарев А.\,В.} % Автор документа
\date{\today} % Текущая дата

\begin{document} % Конец преамбулы, начало текста

\maketitle % Печатает заголовок, список авторов и дату

\section{Задание №1}
Реализация взята из Приложения, разобрана и протестирована:
%\lstinputlisting[language=Python, firstline=11, lastline=16]{../code8.py}

В секции Исходный код представлены тесты и ожидаемые результаты на указанной в задание выборке.

\section{Задание №2}
Для простых функций, имеющих реализацию в языке R была реализована функция вызова:
%\lstinputlisting[language=Python, firstline=5, lastline=8]{../code8.py}

С помощью функции вызова функции вычисления характеристик дескриптивной статистики реализуются в одну строчку:
%\lstinputlisting[language=Python, firstline=19, lastline=32]{../code8.py}

Для нахождения моды не получилось найти простого аналога в языке R, поэтому пришлось объявить и вызвать свою реализацию:
%\lstinputlisting[language=Python, firstline=35, lastline=44]{../code8.py}

В секции Исходный код представлены тесты и ожидаемые результаты на указанной в задание выборке.

\section{Задание №3}
Для аналога доверительного интервала в R не нашёл способа получить половину длины интервала, поэтому в Python взял правую границу и вычел среднее выборочное.

%\lstinputlisting[language=Python, firstline=47, lastline=50]{../code8.py}

В секции Исходный код представлен тест линейной регрессии с фиксированными значениями псевдо-генеротора чисел.

\section{Задание №4}
Выборки были сгенерированы внутри R, там же была вызвана функция lm, из вывода которой были получены коэффициенты.
\newpage
%\lstinputlisting[language=Python, firstline=53, lastline=62]{../code8.py}

\section{Задание №5}
Код R считывается из файла в Python и затем запускается:
%\lstinputlisting[language=Python, firstline=65, lastline=72]{../code8.py}
\newpage
Выборки были сгенерированы внутри R, там же была вызвана функция lm, построен график выборок и линейной регрессии и затем сохранён в виде png файла:

%\noindent\makebox[\textwidth]{\includegraphics[width=0.7\paperwidth]{figure_1}}

\section{Пояснение}
Исходный код доступен по ссылке:

\href{https://github.com/SvichkarevAnatoly/Course-Python-Bioinformatics/tree/master/bioseq7}{https://github.com/SvichkarevAnatoly/Course-Python-Bioinformatics/tree/master/bioseq8}

\section{Исходный код}
Файл \verb$test_code8.py$:
%\lstinputlisting[language=Python]{../test_code8.py}

Файл \verb$r_lr_plot.R$:
%\lstinputlisting[language=Python]{../r_lr_plot.R}

\end{document} % Конец документа
