\documentclass{article} % Класс печатного документа

% для поддержки русского языка
\usepackage[T2A]{fontenc} % поддержка специальных русских символов
\usepackage[utf8]{inputenc} % Кодировка исходного текста - utf8
\usepackage[english,russian]{babel} % Поддержка языка - русского с английским
\usepackage{indentfirst} % Отступ в первом абзаце

\usepackage{hyperref} % Для вставки гиперссылок
\usepackage{listings} % Для вставки кусков кода
\usepackage{graphicx} % Вставка изображений
\usepackage{subfig} % Изображения друг напротив друга
\usepackage{float} % Для точного позиционирования картинок

\title{Отчёт 3\protect\\Кластерный анализ микрочипов} % Заголовок документа
\author{Свичкарев А.\,В.} % Автор документа
\date{\today} % Текущая дата

\begin{document} % Конец преамбулы, начало текста

\maketitle % Печатает заголовок, список авторов и дату

\section{Цель}
Изучить способы решения задач
поиска кластеров в данных микрочипов

\section{Задание №1}
Написать программу чтения
изображения микрочипа «RedGreenArray.png» и
сохранеия матрицы данных в виде файла
(строка, столбец, канал1, канал2,канал3).

Реализация взята из Приложения.

Начальная часть файла \verb$RedGreenArrayData.txt$:
% \lstinputlisting[lastline=20]{RedGreenArrayData.txt}

\clearpage
\section{Задание №2}
Написать программу иерархической кластеризации
строк матрицы данных из предыдущего задания.
Для вычисления метрики расстояния различия
между двумя канальными интенсивностями
(RED, GREEN) использовать gScore.
\bigskip

Реализация взята из Приложения.
\bigskip

Изображение микрочипа,
построенное по матрице
из предыдущего задания
в градациях серого:
\bigskip

% \includegraphics{im-11.png}

\clearpage
Иерархическая кластеризация,
построенная по двум каналам
интенсивности с использованием
метрики gScore:
\bigskip

% \includegraphics{im-12.png}

\clearpage
\section{Выводы}
Изучены способы решения задач поиска кластеров
в данных микрочипов.
Освоена работа с растровыми изображениями,
извлечение из них данных каналов.
Сохранение результатов в текстовую таблицу
и кластеризация матрицы антител.

\section{Пояснение}
Исходный код доступен по ссылке:
\href{https://github.com/SvichkarevAnatoly/Course-Python-Bioinformatics/tree/master/semester2/task4}
{github.com}

\section{Исходный код}
Файл \verb$ArrayData.py$:
% \lstinputlisting[language=Python]{../ArrayData.py}

\end{document} % Конец документа
