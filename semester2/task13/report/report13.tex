\documentclass{article} % Класс печатного документа

% для поддержки русского языка
\usepackage[T2A]{fontenc} % поддержка специальных русских символов
\usepackage[utf8]{inputenc} % Кодировка исходного текста - utf8
\usepackage[english,russian]{babel} % Поддержка языка - русского с английским
\usepackage{indentfirst} % Отступ в первом абзаце

\usepackage{hyperref} % Для вставки гиперссылок
% \usepackage{listings} % Для вставки кусков кода
\usepackage{graphicx} % Вставка изображений
\usepackage{subfig} % Изображения друг напротив друга
\usepackage{float} % Для точного позиционирования картинок
\usepackage[justification=centering]{caption} % Для центрирования подписей

% Default fixed font does not support bold face
\DeclareFixedFont{\ttb}{T1}{txtt}{bx}{n}{12} % for bold
\DeclareFixedFont{\ttm}{T1}{txtt}{m}{n}{12}  % for normal

% Custom colors
\usepackage{color}
\definecolor{deepblue}{rgb}{0,0,0.5}
\definecolor{deepred}{rgb}{0.6,0,0}
\definecolor{deepgreen}{rgb}{0,0.5,0}

\usepackage{listings}

% Python style for highlighting
\newcommand\pythonstyle{\lstset{
    language=Python,
    basicstyle=\ttm,
    otherkeywords={self},             % Add keywords here
    keywordstyle=\color{deepblue},
    emph={MyClass,__init__},          % Custom highlighting
    emphstyle=\color{deepred},    % Custom highlighting style
    stringstyle=\color{deepgreen},
    frame=tb,                         % Any extra options here
    showstringspaces=false,           % 
    basicstyle=\small,                % уменьшить размер шрифта
    columns=flexible                  % чтобы при копировании не было пробелов везде
}}


% Python environment
\lstnewenvironment{python}[1][]
{
\pythonstyle
\lstset{#1}
}
{}

% Python for external files
\newcommand\pythonexternal[2][]{{
\pythonstyle
\lstinputlisting[#1]{#2}}}

% Python for inline
\newcommand\pythoninline[1]{{\pythonstyle\lstinline!#1!}}

\makeatletter
\def\lst@outputspace{{\ifx\lst@bkgcolor\empty\color{white}\else\lst@bkgcolor\fi\lst@visiblespace}}
\makeatother
 % для красивого оформления python кода

% путь к папке с изображениями
\graphicspath{{./figs/}}

\title{Отчёт 13\\
Метод байесовской классификации (NB).
Сравнение изученных методов классификации} % заголовок документа
\author{Свичкарев А.\,В.} % Автор документа
\date{\today} % Текущая дата

\begin{document} % Конец преамбулы, начало текста

\maketitle % Печатает заголовок, список авторов и дату

\section{Цель}
Изучить способы решения задач классификации методом Naive
Bayes (NB) и сравнить его эффективность с эффективностью изученных ранее методов.

\section{Задание №1}
Написать программу построения
модели Байесовской классификации (NB)
с использованием кода из Приложений 1-2.
В качестве входных использовать данные
из Примера №1.
Оценить точность предсказания,
затраченное время и охарактеризовать
полученные результаты.

Реализация на основе кода программы из Приложения.
\pythonexternal{../exercise1.py}
\bigskip

Вывод предсказания двух классов семантики
по фразе, содержащей слова из тренировочной последовательности.
Классификация проводилась на основе Байесовской модели.
Приведены оценки точности предсказания.
\lstinputlisting{ex1_out.txt}

\clearpage
\section{Задание №2}
Написать программу классификации
с использованием метода GaussianNB
из библиотеки sklearn (см. Приложение 3).
Применить программу к данным из Примера №2.
Оценить точность, время
и охарактеризовать полученные результаты
в сравнении с другими ранее изученными методами
(KNeighborsClassifier, DecisionTreeClassifier,
RandomForestClassifier, AdaBoostClassifier,
SVC c линейным и RBF ядрами).

Реализация на основе кода программы из Приложения.
\pythonexternal{../exercise2.py}

К отчёту прилагается сводная диаграмма
для трёх выборок
протестированная на выбранных классификаторах.
Полупрозрачным цветом выделены
элементы тестовой выборки.
В правом нижнем углу каждого графика
показана точность на тестовой выборки.

\section{Выводы}
SVM с линейным ядром показал себя хуже всего на данных выборках,
так как их нельзя назвать линейно разделяемыми в пространстве признаков.

Однако применение гауссова ядра показало лучший среди остальных
сравниваемых классификаторов результат,
так как оно позволяет построить гиперплоскость
в расширенном пространстве признаков,
которое больше по размерности, чем исходное.

Дерево принятия решений
при достаточном значение глубины
способно показывать достойные результаты.
Результирующее дерево легко для понимания
работы классификатора.

В общем случае алгоритм random forest
показывает лучшие результаты,
чем дерево принятия решений,
так как по сути использует
голосование большого числа
деревьев принятия решений,
построенных различными способами.
Почему на третьей выборке
точность оказалась хуже,
чем для обычного дерева принятия решений
не совсем понятно.

Алгоритм k-ближайших соседей хорошо
работает для связных выборок сложной
геометрической формы,
так как опирается на близость элементов
одного класса. Однако, если
классы зашумлены и смешаны,
данный алгоритм начинает проигрывать.
Для данного алгоритма так же нужно знать
число классов, что в общем случае может быть не известно.

Байесов классификатор чем-то схож
с SVM и использует статистическую модель.
Показал средние результаты.

\section{Пояснение}
Исходный код доступен по ссылке:
\href{https://github.com/SvichkarevAnatoly/Course-Python-Bioinformatics/tree/master/semester2/task13}
{github.com}

\end{document} % Конец документа
