\documentclass{article} % Класс печатного документа

\usepackage{hyperref} % Для вставки гиперссылок
\usepackage{listings} % Для вставки кусков кода
\usepackage{graphicx} % Вставка изображений
\usepackage[utf8]{inputenc} % Кодировка исходного текста - utf8
\usepackage[english,russian]{babel} % Поддержка языка - русского с английским
\usepackage{indentfirst} % Отступ в первом абзаце

\title{Отчёт 5\protect\\Нейронные сети} % Заголовок документа
\author{Свичкарев А.\,В.} % Автор документа
\date{\today} % Текущая дата

\begin{document} % Конец преамбулы, начало текста

\maketitle % Печатает заголовок, список авторов и дату

\section{Задание №1}
Функция активации:

TODO

\section{Задание №2}

\section{Задание №3}

\section{Пояснение}
За основу был взят и разобран код программы из Приложения:
\begin{itemize}
	\item Удалены лишние переменные:
		\begin{lstlisting}[language=Python]
sigInp = ones(numInp)
sigHid = ones(numHid)
sigOut = ones(numOut)
		\end{lstlisting}
	\item Удалил излишнее преобразование к array:
		\begin{lstlisting}[language=Python]
trainData[x] = (array(inputs), array(knownOut))
		\end{lstlisting}
	\item Изменил инициализацию поиска минимальной ошибки:
		\begin{lstlisting}[language=Python]
minError = sys.float_info.max
		\end{lstlisting}
вместо None
	\item Привел к более лаконичному варианту (вместо цикла for):
		\begin{lstlisting}[language=Python]
ssIndexDict = {x: i for i, x in enumerate(ssCodes)}
		\end{lstlisting}
\end{itemize}

\section{Исходный код}
% \lstinputlisting[language=Python]{../code5.py}

\end{document} % Конец документа
