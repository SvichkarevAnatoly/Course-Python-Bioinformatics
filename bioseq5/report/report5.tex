\documentclass{article} % Класс печатного документа

\usepackage{hyperref} % Для вставки гиперссылок
\usepackage{listings} % Для вставки кусков кода
\usepackage{graphicx} % Вставка изображений
\usepackage[utf8]{inputenc} % Кодировка исходного текста - utf8
\usepackage[english,russian]{babel} % Поддержка языка - русского с английским
\usepackage{indentfirst} % Отступ в первом абзаце

\title{Отчёт 5\protect\\Нейронные сети} % Заголовок документа
\author{Свичкарев А.\,В.} % Автор документа
\date{\today} % Текущая дата

\begin{document} % Конец преамбулы, начало текста

\maketitle % Печатает заголовок, список авторов и дату

\section{Цель работы}
Изучить способ решения задачи предсказания биологических последовательностей с использованием нейронных сетей.

\section{Задание №1}
Функция активации - гиперболический тангенс:
\[f(s)=th\frac{s}{\alpha}=\frac{e^\frac{s}{\alpha}-e^{-\frac{s}{\alpha}}}{e^\frac{s}{\alpha}+e^{-\frac{s}{\alpha}}}\]

Для обучения использовалась выборка из 16 пятисимвольных аминокислотных последовательностей с известными классами вторичной структуры.

Значения выходных сигналов сети:
\begin{verbatim}
[ 0.94314467  0.01409836  0.86001628]
\end{verbatim}
для классов вторичной структуры белка H, C, E соответственно.

Как видно из значений, последовательность DLLSA принадлежит к \mbox{классу} H.

\section{Задание №2}

Обучение происходило на значительно большей выборке из 26242 последовательностей. 

Значения выходных сигналов сети:
\begin{verbatim}
[-0.30372237 -0.44456174  0.86245037]
\end{verbatim}
для классов вторичной структуры белка H, C, E соответственно.

Нейронная сеть на данной обучающей выборке и текущей последовательности псевдослучайных чисел приписала последовательности DLLSA класс E.

\section{Задание №3}
Функция активации - логистическая:
\[f(s)=\frac{1}{1+e^{-s}}\]

Для выборки из предыдущего задания получились следующие выходные сигналы:
\begin{verbatim}
[ nan  nan  nan]
\end{verbatim}

Причина скорее всего заключается в менее подходящей функции активации для данной выборки, где-то происходит переполнение.

Дополнительно получены выходные сигналы для обучающей выборки из задания 1 с той же функцией активации:
\begin{verbatim}
[ 0.5  0.5  0.5]
\end{verbatim}

В данном случае мы не можем сказать к какому классу относится последовательность, выходные сигналы равны для всех трёх классов.

\section{Пояснение}
За основу был взят и разобран код программы из Приложения, главные изменения:
\begin{itemize}
	\item Выделение классa нейронной сети
	\item Добавление объекта генератора псевдослучайных чисел для объекта нейронной сети для детерминированности поведения нейронной сети при тестировании
	\item Удаление лишние переменные:
		\begin{lstlisting}[language=Python]
sigInp = ones(numInp)
sigHid = ones(numHid)
sigOut = ones(numOut)
		\end{lstlisting}
	\item Удаление излишних преобразований к array:
		\begin{lstlisting}[language=Python]
trainData[x] = (array(inputs), array(knownOut))
		\end{lstlisting}
	\item Изменение инициализации поиска минимальной ошибки:
		\begin{lstlisting}[language=Python]
minError = sys.float_info.max
		\end{lstlisting}
вместо None
	\item Приведение к более лаконичному варианту (вместо цикла for):
		\begin{lstlisting}[language=Python]
ssIndexDict = {x: i for i, x in enumerate(ssCodes)}
		\end{lstlisting}
\end{itemize}

\section{Исходный код}
Приводится код только класса нейронной сети. Код тестов и код запуска на большой выборке приведен на \href{https://github.com/SvichkarevAnatoly/Course-Python-Bioinformatics/tree/master/bioseq5}{github.com}.

\lstinputlisting[language=Python]{../code5.py}

\end{document} % Конец документа
