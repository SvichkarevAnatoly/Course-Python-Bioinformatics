\documentclass{article} % Класс печатного документа

\usepackage{hyperref} % Для вставки гиперссылок
\usepackage{listings} % Для вставки кусков кода
\usepackage{graphicx} % Вставка изображений
\usepackage[utf8]{inputenc} % Кодировка исходного текста - utf8
\usepackage[english,russian]{babel} % Поддержка языка - русского с английским
\usepackage{indentfirst} % Отступ в первом абзаце

\title{Отчёт 9\protect\\Параллельные вычисления при решении задач} % Заголовок документа
\author{Свичкарев А.\,В.} % Автор документа
\date{\today} % Текущая дата

\begin{document} % Конец преамбулы, начало текста

\maketitle % Печатает заголовок, список авторов и дату

\section{Задание №1}
Для реализации данной задачи использовался пул потоков.
Было сгенерировано 9 выборок из стандартного нормального распределения и добавлена исходная выборка. 
Для каждой выборки добавлена в пул потоков задача вычисления t-теста из предыдущей лабораторной работы.
После завершения работы всех задач собираются результаты и печатаются. 

\lstinputlisting[language=Python]{../code9_task1.py}

Вывод программы:
\begin{verbatim}
t_test 0:   t=-2.07215145423    p-value=0.0654032989892
t_test 1:   t=1.07698200298 p-value=0.314386172786
t_test 2:   t=1.5280440573  p-value=0.170209043555
t_test 3:   t=1.54003341329 p-value=0.159761269603
t_test 4:   t=0.105582502203    p-value=0.91807398696
t_test 5:   t=-1.89195960342    p-value=0.0957531727758
t_test 6:   t=-0.977784941835   p-value=0.353715362256
t_test 7:   t=-0.0746585931577  p-value=0.941962594609
t_test 8:   t=0.418806352869    p-value=0.687051129944
t_test 9:   t=0.817795153197    p-value=0.432601843833
\end{verbatim}

\newpage
\section{Задание №2}
Для данной задачи использовалась очередь для сохранения результатов работы программы.
Сгенерировано 9 выборок и одна выборка из задания.
Для каждой выборки в параллель запускаются все функции дескриптивной статистики.
После завершения каждого потока из очереди собирается результаты и выводятся на экран.
Для вызова различных функций была написана обёртка с передачей вызываемой функции дескриптивной статистики и выборки.

\lstinputlisting[language=Python]{../code9_task2.py}

\newpage
Вывод программы:
\begin{verbatim}
On [1, 2, 2, 3, 2, 1, 4, 2, 3, 1, 0] mode = 2
On [1, 2, 2, 3, 2, 1, 4, 2, 3, 1, 0] mean = 1.90909090909
On [1, 2, 2, 3, 2, 1, 4, 2, 3, 1, 0] dispersion = 1.08330684435
On [1, 2, 2, 3, 2, 1, 4, 2, 3, 1, 0] standard_deviation = 1.17355371901
On [1, 2, 2, 3, 2, 1, 4, 2, 3, 1, 0] median = 2.0
On [4, 3, 2, 1, 2, 2, 3, 1, 2, 2, 4] mean = 2.36363636364
On [4, 3, 2, 1, 2, 2, 3, 1, 2, 2, 4] mode = 2
On [4, 3, 2, 1, 2, 2, 3, 1, 2, 2, 4] standard_deviation = 0.95867768595
On [4, 3, 2, 1, 2, 2, 3, 1, 2, 2, 4] dispersion = 0.979120874024
On [4, 3, 2, 1, 2, 2, 3, 1, 2, 2, 4] median = 2.0
On [2, 1, 3, 3, 1, 4, 4, 4, 4, 1, 3] mean = 2.72727272727
On [2, 1, 3, 3, 1, 4, 4, 4, 4, 1, 3] mode = 4
On [2, 1, 3, 3, 1, 4, 4, 4, 4, 1, 3] median = 3.0
On [2, 1, 3, 3, 1, 4, 4, 4, 4, 1, 3] standard_deviation = 1.47107438017
On [2, 1, 3, 3, 1, 4, 4, 4, 4, 1, 3] dispersion = 1.21287855128
On [4, 3, 2, 0, 2, 3, 4, 4, 2, 4, 1] mean = 2.63636363636
On [4, 3, 2, 0, 2, 3, 4, 4, 2, 4, 1] mode = 4
On [4, 3, 2, 0, 2, 3, 4, 4, 2, 4, 1] dispersion = 1.29844153246
On [4, 3, 2, 0, 2, 3, 4, 4, 2, 4, 1] standard_deviation = 1.68595041322
On [4, 3, 2, 0, 2, 3, 4, 4, 2, 4, 1] median = 3.0
On [4, 2, 0, 3, 1, 4, 3, 0, 2, 4, 1] mean = 2.18181818182
On [4, 2, 0, 3, 1, 4, 3, 0, 2, 4, 1] median = 2.0
On [4, 2, 0, 3, 1, 4, 3, 0, 2, 4, 1] standard_deviation = 2.14876033058
On [4, 2, 0, 3, 1, 4, 3, 0, 2, 4, 1] mode = 4
On [4, 2, 0, 3, 1, 4, 3, 0, 2, 4, 1] dispersion = 1.46586504515
On [1, 4, 0, 2, 1, 4, 4, 2, 0, 1, 2] mean = 1.90909090909
On [1, 4, 0, 2, 1, 4, 4, 2, 0, 1, 2] mode = 1
On [1, 4, 0, 2, 1, 4, 4, 2, 0, 1, 2] median = 2.0
On [1, 4, 0, 2, 1, 4, 4, 2, 0, 1, 2] dispersion = 1.44313707876
On [1, 4, 0, 2, 1, 4, 4, 2, 0, 1, 2] standard_deviation = 2.0826446281
On [4, 0, 2, 3, 2, 4, 2, 4, 3, 2, 2] mean = 2.54545454545
On [4, 0, 2, 3, 2, 4, 2, 4, 3, 2, 2] mode = 2
On [4, 0, 2, 3, 2, 4, 2, 4, 3, 2, 2] dispersion = 1.15708382376
On [4, 0, 2, 3, 2, 4, 2, 4, 3, 2, 2] standard_deviation = 1.33884297521
On [4, 0, 2, 3, 2, 4, 2, 4, 3, 2, 2] median = 2.0
On [2, 1, 2, 1, 0, 0, 3, 3, 2, 0, 3] mode = 0
On [2, 1, 2, 1, 0, 0, 3, 3, 2, 0, 3] mean = 1.54545454545
On [2, 1, 2, 1, 0, 0, 3, 3, 2, 0, 3] median = 2.0
On [2, 1, 2, 1, 0, 0, 3, 3, 2, 0, 3] dispersion = 1.15708382376
On [2, 1, 2, 1, 0, 0, 3, 3, 2, 0, 3] standard_deviation = 1.33884297521
On [4, 4, 4, 4, 4, 2, 1, 3, 1, 4, 4] mean = 3.18181818182
On [4, 4, 4, 4, 4, 2, 1, 3, 1, 4, 4] median = 4.0
On [4, 4, 4, 4, 4, 2, 1, 3, 1, 4, 4] dispersion = 1.19226154987
On [4, 4, 4, 4, 4, 2, 1, 3, 1, 4, 4] mode = 4
On [4, 4, 4, 4, 4, 2, 1, 3, 1, 4, 4] standard_deviation = 1.42148760331
On [4, 2, 4, 2, 2, 3, 4, 4, 3, 0, 3] mean = 2.81818181818
On [4, 2, 4, 2, 2, 3, 4, 4, 3, 0, 3] mode = 4
On [4, 2, 4, 2, 2, 3, 4, 4, 3, 0, 3] dispersion = 1.19226154987
On [4, 2, 4, 2, 2, 3, 4, 4, 3, 0, 3] standard_deviation = 1.42148760331
On [4, 2, 4, 2, 2, 3, 4, 4, 3, 0, 3] median = 3.0
\end{verbatim}

\newpage
\section{Задание №3}
Для данной задачи были сгенерированны указанные выборки и запущены одновременно три потока для вычисления линейной регрессии для каждой выборки. Потоки сразу после вычисления выводят результат на экран.

\lstinputlisting[language=Python]{../code9_task3.py}

Вывод программы:
\begin{verbatim}
on -0.7 sample: grad = -0.677060313291 yInt = 2.01503061701
on -2.7 sample: grad = -2.67528441296 yInt = 1.96210461878
on -1.7 sample: grad = -1.68151210834 yInt = 1.95964608988
\end{verbatim}


\section{Пояснение}
Исходный код доступен по ссылке:

\href{https://github.com/SvichkarevAnatoly/Course-Python-Bioinformatics/tree/master/bioseq9}{https://github.com/SvichkarevAnatoly/Course-Python-Bioinformatics/tree/master/bioseq9}

\end{document} % Конец документа
